%##############################################################################
% Packages
%##############################################################################

%<*tag1>
\usepackage{amsfonts,amssymb}                   %Standard math packages.
\usepackage{centernot,enumerate,mathrsfs}       %Not-divides symbol; enumerate arguments;\mathscr font.
\usepackage{color}                              %Colors.
\usepackage[usenames,dvipsnames,svgnames,table]{xcolor} %More colors!
\usepackage{hyperref}                           %Linkable items.
%\usepackage[utf8x]{inputenc}                    %Support for some unicode (e.g. emdash) without using XeTeX.
\usepackage{tikz}                               %Create graphics.
\usepackage{cancel}
%\usepackage{tikz-cd}                            %Commutative diagrams.
\usetikzlibrary{patterns,shapes,snakes}
\usepackage{graphicx}                           %Import graphics.
\usepackage{listings}

% Use knitr's colorscheme.
\definecolor{fgcolor}{rgb}{0.345, 0.345, 0.345}
\definecolor{hlnum}{rgb}{0.686,0.059,0.569}
\definecolor{hlstr}{rgb}{0.192,0.494,0.8}
\definecolor{hlcom}{rgb}{0.678,0.584,0.686}
\definecolor{hlopt}{rgb}{0,0,0}
\definecolor{hlstd}{rgb}{0.345,0.345,0.345}
\definecolor{hlkwa}{rgb}{0.161,0.373,0.58}
\definecolor{hlkwb}{rgb}{0.69,0.353,0.396}
\definecolor{hlkwc}{rgb}{0.333,0.667,0.333}
\definecolor{hlkwd}{rgb}{0.737,0.353,0.396}
\definecolor{shadecolor}{rgb}{0.969, 0.969, 0.969}

%\usepackage{fontspec, xunicode}
%\setmonofont{Consolas}

\lstset{
  backgroundcolor=\color{shadecolor},
  basicstyle=\color{hlstd}\ttfamily\footnotesize,
  breakatwhitespace=false,
  %breaklines=true,
  captionpos=b,
  commentstyle=\color{hlcom},
  deletekeywords={...},
  escapeinside={\%*}{*)},
  extendedchars=true,
  frame=lines,
  keepspaces=true,
  keywordstyle=\color{hlkwb},
  morekeywords={*,...},
  numbers=left,
  numbersep=5pt,
  numberstyle=\tiny\color{hlstd},
  rulecolor=\color{hlstd},
  showspaces=false,
  showstringspaces=false,
  showtabs=false,
  stepnumber=1,
  stringstyle=\color{hlstr},
  tabsize=2,
  title=\lstname
}
%</tag1>
\usepackage{titlesec}                           %Reformats sections.

%##############################################################################
% Font, Margins, & Spacing
%##############################################################################

\renewcommand{\familydefault}{\sfdefault}       %Font
\parindent0em
\parskip1em
\allowdisplaybreaks

%##############################################################################
% Ordering of Paper
%##############################################################################

\numberwithin{equation}{section}    %Equation counter.
\newtheorem{thm}{Theorem}[section]  %Theorem Counter.
                        %Section & Subsection Counter.
\newcommand{\twodigits}[1]{\ifnum#1<10 0#1\else #1\fi}  %Appends 0's.
\newcommand{\sectionbreak}{\clearpage}      %Breaks sections to pages.

\titleformat{\section}[display]{\scshape\fillast}{Lecture \oldstylenums{\twodigits{\thesection}}}{-.5em}{\Large}
\titleformat{\subsection}{\scshape\large}{\S\thesubsection}{0.5em}{}
\titlespacing*{\section}{3pc}{*3}{*2}[3pc]
%Issues:
%       HEADERS: Set only page number for first lecture-page, alternate author name and lecture title after.
%       DATE: Set footnote on title, without number, but allow for footnotes with numbers in general.
%       TABLE OF CONTENTS: general formatting, hyperlinks.
%       SECTION TITLE FORMAT: Do I want to bold the section titles , as it normally is?
%Wish List:
%Custom TOC (like GITPreamble's; hyperlinks, general formatting).
%Custom \section commands (titlesec package).

%##############################################################################
% Typefont Macros
%##############################################################################

%<*tag2>
%\renewcommand{\b}{\mathbb}          %Special sets.
%\renewcommand{\c}{\mathcal}         %Schemes, categories, cryptography, logic.
%\newcommand{\f}{\mathfrak}          %Ideals, Lie algebras.
%\newcommand{\s}{\mathscr}              %Alternative: topological spaces, algebras, sheaves, categories.

%\renewcommand{\tt}{\text}            %Spacings: ~,\quad,\qquad,\hspace.
%\renewcommand{\em}[1]{\it{\und{#1}}} %Keywords.
%\renewcommand{\bf}{\textbf}          %Vectors.
%\renewcommand{\it}{\textit}
%\renewcommand{\sc}{\textsc}
%\newcommand{\ttt}{\texttt}
%\newcommand{\und}{\underline}


%\renewcommand{\hat}{\widehat}
%\renewcommand{\tilde}{\widetilde}
%\renewcommand{\bar}{\overline}
\makeatletter
\def\footnotenonum{\xdef\@thefnmark{}\@footnotetext}
\makeatother
\newcommand{\circled}[1]{\raisebox{.5pt}{\textcircled{\raisebox{-.9pt}{#1}}}}
%\newcommand{\xym}{\xymatrix}

%\newcommand{\bA}{\b A}   \newcommand{\bB}{\b B}   \newcommand{\bC}{\b C}
%\newcommand{\bD}{\b D}   \newcommand{\bE}{\b E}   \newcommand{\bF}{\b F}
%\newcommand{\bG}{\b G}   \newcommand{\bH}{\b H}   \newcommand{\bI}{\b I}
%\newcommand{\bJ}{\b J}   \newcommand{\bK}{\b K}   \newcommand{\bL}{\b L}
%\newcommand{\bM}{\b M}   \newcommand{\bN}{\b N}   \newcommand{\bO}{\b O}
%\newcommand{\bP}{\b P}   \newcommand{\bQ}{\b Q}   \newcommand{\bR}{\b R}
%\newcommand{\bS}{\b S}   \newcommand{\bT}{\b T}   \newcommand{\bU}{\b U}
%\newcommand{\bV}{\b V}   \newcommand{\bW}{\b W}   \newcommand{\bX}{\b X}
%\newcommand{\bY}{\b Y}   \newcommand{\bZ}{\b Z}

%\newcommand{\cA}{\c A}   \newcommand{\cB}{\c B}   \newcommand{\cC}{\c C}
%\newcommand{\cD}{\c D}   \newcommand{\cE}{\c E}   \newcommand{\cF}{\c F}
%\newcommand{\cG}{\c G}   \newcommand{\cH}{\c H}   \newcommand{\cI}{\c I}
%\newcommand{\cJ}{\c J}   \newcommand{\cK}{\c K}   \newcommand{\cL}{\c L}
%\newcommand{\cM}{\c M}   \newcommand{\cN}{\c N}   \newcommand{\cO}{\c O}
%\newcommand{\cP}{\c P}   \newcommand{\cQ}{\c Q}   \newcommand{\cR}{\c R}
%\newcommand{\cS}{\c S}   \newcommand{\cT}{\c T}   \newcommand{\cU}{\c U}
%\newcommand{\cV}{\c V}   \newcommand{\cW}{\c W}   \newcommand{\cX}{\c X}
%\newcommand{\cY}{\c Y}   \newcommand{\cZ}{\c Z}

%\newcommand{\fa}{\f a}   \newcommand{\fb}{\f b}   \newcommand{\fc}{\f c}
%\newcommand{\fd}{\f d}   \newcommand{\fe}{\f e}   \newcommand{\ff}{\f f}
%\newcommand{\fg}{\f g}   \newcommand{\fh}{\f h}   \newcommand{\Fi}{\f i}
%\newcommand{\fj}{\f j}   \newcommand{\fk}{\f k}   \newcommand{\fl}{\f l}
%\newcommand{\fm}{\f m}   \newcommand{\fn}{\f n}   \newcommand{\fo}{\f o}
%\newcommand{\fp}{\f p}   \newcommand{\fq}{\f q}   \newcommand{\fr}{\f r}
%\newcommand{\fs}{\f s}   \newcommand{\ft}{\f t}   \newcommand{\fu}{\f u}
%\newcommand{\fv}{\f v}   \newcommand{\fw}{\f w}   \newcommand{\fx}{\f x}
%\newcommand{\fy}{\f y}   \newcommand{\fz}{\f z}

%\newcommand{\sA}{\s A}   \newcommand{\sB}{\s B}   \newcommand{\sC}{\s C}
%\newcommand{\sD}{\s D}   \newcommand{\sE}{\s E}   \newcommand{\sF}{\s F}
%\newcommand{\sG}{\s G}   \newcommand{\sH}{\s H}   \newcommand{\sI}{\s I}
%\newcommand{\sJ}{\s J}   \newcommand{\sK}{\s K}   \newcommand{\sL}{\s L}
%\newcommand{\sM}{\s M}   \newcommand{\sN}{\s N}   \newcommand{\sO}{\s O}
%\newcommand{\sP}{\s P}   \newcommand{\sQ}{\s Q}   \newcommand{\sR}{\s R}
%\newcommand{\sS}{\s S}   \newcommand{\sT}{\s T}   \newcommand{\sU}{\s U}
%\newcommand{\sV}{\s V}   \newcommand{\sW}{\s W}   \newcommand{\sX}{\s X}
%\newcommand{\sY}{\s Y}   \newcommand{\sZ}{\s Z}

%##############################################################################
% Operator Macros
%##############################################################################

\newcommand{\Ob}{\operatorname{Ob}}                        %Set-inducing operators.
\newcommand{\Arr}{\operatorname{Arr}}
\newcommand{\Ker}{\operatorname{Ker}}
\newcommand{\Coker}{\operatorname{Coker}}
\newcommand{\Coim}{\operatorname{Coim}}
\newcommand{\Hom}{\operatorname{Hom}}
\newcommand{\Isom}{\operatorname{Isom}}
\newcommand{\End}{\operatorname{End}}
\newcommand{\Tor}{\operatorname{Tor}}
\newcommand{\Gal}{\operatorname{Gal}}
\newcommand{\Spec}{\operatorname{Spec}}
\newcommand{\Proj}{\operatorname{Proj}}
\newcommand{\Pic}{\operatorname{Pic}}
\newcommand{\Cone}{\operatorname{Cone}}
\newcommand{\Aut}{\operatorname{Aut}}
\newcommand{\Inn}{\operatorname{Inn}}
\newcommand{\Sym}{\operatorname{Sym}}
\newcommand{\Ext}{\operatorname{Ext}}
\newcommand{\Sub}{\operatorname{Sub}}
\newcommand{\Id}{\operatorname{Id}}
\newcommand{\id}{\operatorname{id}}

\newcommand{\del}{\nabla}                       %Arithmetic operators.
\newcommand{\divv}{\operatorname{div}}
\newcommand{\grad}{\operatorname{grad}}
\newcommand{\curl}{\operatorname{curl}}
\newcommand{\vol}{\operatorname{vol}}
\newcommand{\ord}{\operatorname{ord}}
\newcommand{\supp}{\operatorname{supp}}
\newcommand{\pfrac}[2]{\frac{\partial #1}{\partial #2}}
\newcommand{\Res}{\operatorname*{Res}}
\renewcommand{\Re}{\operatorname{Re}}
\renewcommand{\Im}{\operatorname{Im}}
\newcommand{\rank}{\operatorname{rank}}
\newcommand{\trace}{\operatorname{trace}}
\newcommand{\rk}{\operatorname{rk}}
\newcommand{\tr}{\operatorname{tr}}
\newcommand{\sign}{\operatorname{sign}}
\newcommand{\sgn}{\operatorname{sgn}}
\newcommand{\rad}{\operatorname{rad}}
\newcommand{\Nil}{\operatorname{Nil}}
\newcommand{\spn}{\operatorname{span}}
\newcommand{\charr}{\operatorname{char}}
\newcommand{\disc}{\operatorname{disc}}
\newcommand{\supr}{\operatorname*{sup}}
\newcommand{\Var}{\operatorname{Var}}
\newcommand{\sd}{\operatorname{sd}}
\newcommand{\se}{\operatorname{se}}
\newcommand{\SD}{\operatorname{SD}}
\newcommand{\Cov}{\operatorname{Cov}}
\DeclareMathOperator*{\argmin}{arg\,min}
\DeclareMathOperator*{\argmax}{arg\,max}

\newcommand{\given[1]}{\:#1\vert\:}

%\newcommand{\sub}{\subset}                      %Set-theoretic relations.
%\renewcommand{\sup}{\supset}
%\newcommand{\sube}{\subseteq}
%\newcommand{\supe}{\supseteq}
\newcommand{\rcompact}{\Subset}

%\renewcommand{\div}{\mid}                       %Misc.
\newcommand{\notdiv}{\centernot|}
\newcommand{\bs}{\backslash}
\newcommand{\tbs}{\textbackslash}
\newcommand{\semitimes}{\rtimes}
\newcommand{\vequal}{\scriptscriptstyle\|}
\newcommand{\numberthis}{\addtocounter{equation}{1}\tag{\theequation}}

%\renewcommand{\to}{\longrightarrow}             %Arrows.
\newcommand{\ra}{\longrightarrow}             %Arrows.
\renewcommand{\mapsto}{\longmapsto}
\newcommand{\lmapsto}{\mathrel{\reflectbox{$\longmapsto$}}}
\newcommand{\hla}{\lhook\joinrel\relbar\joinrel\leftarrow}
\newcommand{\hra}{\lhook\joinrel\relbar\joinrel\rightarrow}
\newcommand{\sqla}{\xymatrix{&\ar@{~>}[l]}}
\newcommand{\sqra}{\xymatrix{\ar@{~>}[r]&}}
\newcommand{\complies}{\Longleftarrow}
\newcommand{\lra}{\longleftrightarrow}
\newcommand{\epi}{\relbar\joinrel\twoheadrightarrow}
%\newcommand{\mono}{\relbar\joinrel\rightarrowtail} doesn't work. need the extra bar inside.
\newcommand{\da}{\downarrow}
\newcommand{\sla}{\leftarrow}
\newcommand{\sra}{\rightarrow}

\renewcommand{\abstractname}{Disclaimer}         %Formatting.

%##############################################################################
% Theorem Macros
%##############################################################################

\theoremstyle{plain}                             %Standardized.
\newtheorem{definition}[thm]{Definition}
\newtheorem{conjecture}[thm]{Conjecture}
\newtheorem{problem}[thm]{Problem}
\newtheorem{lemma}[thm]{Lemma}
\newtheorem{proposition}[thm]{Proposition}
\newtheorem{property}[thm]{Property}
\newtheorem{theorem}[thm]{Theorem}
\newtheorem{corollary}[thm]{Corollary}
\newtheorem{example}[thm]{Example}
\newtheorem{examples}[thm]{Examples}
\newtheorem{remark}[thm]{Remark}

\newtheorem{algorithm}[thm]{Algorithm}           %Computational.
\newtheorem{procedure}[thm]{Procedure}

\newtheorem{assumption}[thm]{Assumption}         %Less formal.
\newtheorem*{claim}{Claim}
\newtheorem{fact}[thm]{Fact}
\newtheorem{convention}[thm]{Convention}
%</tag2>

%<*tag3>
\newtheorem{exercise}[thm]{Exercise}             %Interactive.
\newtheorem{question}[thm]{Question}
\theoremstyle{definition}
\newtheorem*{answer}{Answer}
%</tag3>
